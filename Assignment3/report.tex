\documentclass[12pt]{article}%
\usepackage{amsfonts}
\usepackage{fancyhdr}
\usepackage{comment}
\usepackage[a4paper, top=2.5cm, bottom=2.5cm, left=2.2cm, right=2.2cm]%
{geometry}
\usepackage{times}
\usepackage{graphicx}

\usepackage{hyperref}

\usepackage{amssymb}
\usepackage{graphicx}%

\begin{document}

\title{MLT - Assignment 3 }
\author{Ayush Sekhari (12185)}
\date{\today}
\maketitle

\section{Part I and II}
Part I and II were hand solved and are attached with the report. Please scroll down to view them. 
\section{Part III}
\subsection{Part III-a}
\subsubsection{Generating Data}
I used inbuilt runif, rnorm functions to generate the data. 600 points for each class were used. 400 of them were used to train and 100, 100 for validation and test respectively. 
The metadata of the data is :
\begin{enumerate}    
\item Training Data : 1200
\item Validation Data : 300 
\item Test Data : 300
\item Priors = [1/3,1/3,1/3] 
\end{enumerate}


\subsubsection{Calculation of the Initial Model}
Each class is to be represented as a mixture of three gaussians. I used {\bf K-means clustering} on training data for each class with k = 3. So each class data is clustered into 3 groups. These give 3 cluster means and co-variance matrices for each class. 
These lead to 9 mean vectors and 9 covariance matrices. 
I also estimate initial priors for each cluster as the cluster group strength. 

\subsubsection{Estimation of Co-Variance Matrix}
I took the pooled co-variance matrix as the weighted average of the cluster co-variance matrices. The weights used were (priors - 1) in order to ensure unbiased estimator of the co-variance. This is kept constant for the problem. 

\subsubsection{Generic Approach:}
I have an initial model with 9 mean vectors and a SIGMA matrix. I vary the mean vectors locally using various approaches, recompute the priors each time and then use the mixture model to estimate the error on the validation set. The one with the best is used and the errors are reported on the test set. Performance and search results using various search approaches are listed in the table below. 

\begin{table}
\centering
\begin{tabular}{|p{6cm}|c|c|}
\hline
Approach & CV error & Test-Set error \\ \hline
k-means \{training error = 12.333\} & 16.666 & 15.666 \\ \hline
Stochastic Variation & 16.333 & 15.333 \\ \hline
Varying all of the 21 components simultaneously with c = \{-0.5,0,0.5\} & 14 & 17.333 \\ \hline
Varying all of the 21 components simultaneously with c = \{-0.1,0,0.1\} & 15.666 & 16.333\\ \hline
\end{tabular}
\caption{Observed Best errors while searching the space around means using various approaches}
\end{table}

\subsection{Part III-b -- LDA and QDA}
\begin{enumerate}
\item LDA gives the error of 0.136 on the test set
\item QDA gives the error of 0.153 on the test set
\item {\bf LDA performs the best}
\end{enumerate}
\end{document}